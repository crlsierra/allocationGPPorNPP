%% Options for packages loaded elsewhere
%\PassOptionsToPackage{unicode}{hyperref}
%\PassOptionsToPackage{hyphens}{url}
%%
%\documentclass[
%]{article}
%\usepackage{lmodern}
%\usepackage{amssymb,amsmath}
%\usepackage{ifxetex,ifluatex}
%\ifnum 0\ifxetex 1\fi\ifluatex 1\fi=0 % if pdftex
%  \usepackage[T1]{fontenc}
%  \usepackage[utf8]{inputenc}
%  \usepackage{textcomp} % provide euro and other symbols
%\else % if luatex or xetex
%  \usepackage{unicode-math}
%  \defaultfontfeatures{Scale=MatchLowercase}
%  \defaultfontfeatures[\rmfamily]{Ligatures=TeX,Scale=1}
%\fi
%% Use upquote if available, for straight quotes in verbatim environments
%\IfFileExists{upquote.sty}{\usepackage{upquote}}{}
%\IfFileExists{microtype.sty}{% use microtype if available
%  \usepackage[]{microtype}
%  \UseMicrotypeSet[protrusion]{basicmath} % disable protrusion for tt fonts
%}{}
%\makeatletter
%\@ifundefined{KOMAClassName}{% if non-KOMA class
%  \IfFileExists{parskip.sty}{%
%    \usepackage{parskip}
%  }{% else
%    \setlength{\parindent}{0pt}
%    \setlength{\parskip}{6pt plus 2pt minus 1pt}}
%}{% if KOMA class
%  \KOMAoptions{parskip=half}}
%\makeatother
%\usepackage{xcolor}
%\IfFileExists{xurl.sty}{\usepackage{xurl}}{} % add URL line breaks if available
%\IfFileExists{bookmark.sty}{\usepackage{bookmark}}{\usepackage{hyperref}}
%\hypersetup{
%  hidelinks,
%  pdfcreator={LaTeX via pandoc}}
%\urlstyle{same} % disable monospaced font for URLs
%\setlength{\emergencystretch}{3em} % prevent overfull lines
%\providecommand{\tightlist}{%
%  \setlength{\itemsep}{0pt}\setlength{\parskip}{0pt}}
%\setcounter{secnumdepth}{-\maxdimen} % remove section numbering
%
%\author{}
%\date{}
%
%\begin{document}

\emph{This is a well written piece. The authored argued why carbon
allocation should be based on GPP rather than NPP, as often done in most
land surface or ecosystem models. These models assume zero residence
time of respired carbon through autotrophic respiration, which
contradicts the field measurements of the age of respired C from
plants.}

We thank the reviewer for recognizing the relevance of our manuscript.

\emph{However, there are several significant limitations with the
proposed approach: (1) ecosystem GPP cannot be measured directly; where
NPP can be easily measured in the field and with a large number of
observations available globally;}

Measuring GPP in the field is challenging, but so is measuring NPP. In a
comprehensive review on the NPP concept and its quantification, Clark et
al.~(2001, Ecological Application 11:356) stated that `Forest NPP cannot
be measured directly; it must be approached by indirect methods'. These
authors convincingly showed that there is a significant discrepancy
between the concept of NPP and the way it is measured in the field. Most
field studies provide only an approximation to the real value, very
likely with an underestimation. Few studies quantify belowground NPP,
and many only quantify one component of NPP, usually wood biomass
production. Although GPP is also difficult to quantify from field
measurements, eddy-covariance estimates of GPP are providing now a
wealth of data from a large number of ecosystems world wide. Synthesis
efforts such as Fluxnet and Fluxcom provide well-curated data-products
of global GPP. In particular, Fluxcom combines remote sensing
information with eddy-flux data to produce global gridded products of
GPP at high spatial and temporal resolution, which could be of immense
value for modeling studies. MODIS GPP is an additional product that has
been very useful in providing approximations to ecosystem level GPP for
a number of empirical and modeling studies. Therefore, we do not think
that this concern from the reviewer is valid given that both NPP and GPP
are difficult to obtain, and information from both can obtained from the
literature for a large variety of sites.

\emph{(2) autotrophic respiration of individual biomass components are
rarely measured, therefore allocation fractions of GPP to individual
biomass components often are not available at ecosystem scale;}

We agree with the review in that there are currently few observations of
biomass and respiration for individual pools, which are rarely measured
in field studies. However, we do not see this point as a limitation, but
rather as an opportunity for new scientific studies that would combine
pool-specific measurements with model development of respiratory
processes in these pools. We think it is important to highlight the
limitations of current models in our manuscript so new improvements are
developed. This is better than pessimistically conclude that there is
currently not enough data so models cannot be improved.

\emph{(3) to model the age of respired C in an ecosystem or land surface
model using the proposed approach by the authors here, we will need to
represent the total carbon in each biomass component using multiple
pools with different availability for respiration and their responses to
stresses, which will introduce quite a few additional, poorly
constrained model parameters, and additional uncertainties in the model.
On the other hand, most of the respired carbon is less than one-year old
in leaf, and much younger than the mean age of woody carbon in stem. In
my view, the authors should provide a more balanced view of the pros and
cons of GPP-based and NPP-based approaches for carbon allocations, and
their potential limitations.}

Yes, the proposed approach would likely introduce a set of poorly
constrained parameters. But the current approach of allocating carbon
from NPP has also a set of poorly constrained parameters that cannot be
optimized with the new set of data on the age of respired CO2. Our aim
here is not to show what cannot be done with the current models, but
rather to show what can be done in the future with models that are
consistent with new data on respiration from specific pools and their
radiocarbon content. There are large opportunities to develop new
data-assimilation studies testing different allocation approaches based
on GPP, and incorporating new datasets on respiration.

Although we showed that a large proportion of carbon is respired in less
than one year, we also showed that a small proportion is respired
several years after fixation. To understand the transit time of carbon
in ecosystems we need to better understand both, processes occurring at
fast timescales and processes occurring on longer timescales. However,
the current approach of allocating carbon from NPP does not allow us to
study these ranges of timescales.

We understand the concern of the reviewer in that our statements may be
too strong, and we tried to balance our text in the new revised version.
Nevertheless, we think it is important to clearly identify the
limitations of current models, and also the opportunities for future
studies.

\emph{Some detailed comments} \emph{Figure 1. I disagree with the
statement that majority of the models assume ``a constant proportion of
GPP''. I know that ACCESS-ESM, GFDL, NorESM2, BCC are not. Not sure
about other models. Because of the pool-size based approach, there is a
negative feedback between respiration and pools, therefore the ratio of
autotrophic respiration of GPP is rather constant when averaged over a
year globally.}

The reviewer is right on this point. The majority of models do not have
a constant allocation scheme but rather show a relatively constant
proportion when aggregating globally. We modified the text in this
section to decrease emphasis on constant respiration ratios. We also
eliminated Figure 1 for the same reason.\\
Nevertheless, we would like to emphasize that the issue is not that
models have constant allocation coefficients. The issue is the source of
carbon for allocation, NPP rather than GPP. We made changes in the
manuscript to put less emphasis on the constancy of C allocation, and
more emphasis on the source of carbon.

\emph{L121. CABLE does not assume a fixed fraction of GPP being respired
by plants. See Wang et al.~(2012), GEOPHYSICAL RESEARCH LETTERS, VOL.
39, L19405, doi:10.1029/2012GL053461.}

We apologize for this inaccuracy. In our original review, we used an
earlier description of CABLE to determine the type of allocation scheme
(Wang et al.~2010, Biogeosciences 7:221), and overlooked the more recent
model description. We updated now the text based on this more recent
reference.

\emph{P9, the last two lines below Figure 3. ``the median transit time
is 0 yr, because the autotrophic respiration flux, which corresponds to
50\% of GPP..''. That statement may be specific to the model of Emanuel
et al.~(1981) and that model is more than 40-year old! In most land
surface models, Ra/GPP is not constant, particularly at daily or
seasonal time scale. The median transit time would have to be weighted
by the carbon flux, then the median transit time will not be zero!}

The statement is indeed specific to this model, and the specific value
may change for other models. However, we noticed a potential
misunderstanding in this comment. In the majority of models, allocation
is done from NPP and Ra is subtracted immediately from GPP as we show in
the manuscript. It does not matter if the median transit time is flux
corrected because the age of the respired carbon is still zero (i.e.,
age\_C\_in\_Pool * respired\_flux\_from\_pool = 0 *
respired\_flux\_from\_pool = 0). Also, we used here the model of Emanuel
et al (1981) because it is a useful model to make our point clearly. It
has a minimum of complexity to show the differences between allocation
schemes without additional details that would be irrelevant and would
obscure our point with extra complexity. The model may not be
appropriate to make specific numerical predictions, but it is very
useful as a reduced complexity model that helps to express ideas
clearly.

%\end{document}
