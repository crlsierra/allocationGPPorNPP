%§% Options for packages loaded elsewhere
%§\PassOptionsToPackage{unicode}{hyperref}
%§\PassOptionsToPackage{hyphens}{url}
%§%
%§\documentclass[
%§]{article}
%§\usepackage{lmodern}
%§\usepackage{amssymb,amsmath}
%§\usepackage{ifxetex,ifluatex}
%§\ifnum 0\ifxetex 1\fi\ifluatex 1\fi=0 % if pdftex
%§  \usepackage[T1]{fontenc}
%§  \usepackage[utf8]{inputenc}
%§  \usepackage{textcomp} % provide euro and other symbols
%§\else % if luatex or xetex
%§  \usepackage{unicode-math}
%§  \defaultfontfeatures{Scale=MatchLowercase}
%§  \defaultfontfeatures[\rmfamily]{Ligatures=TeX,Scale=1}
%§\fi
%§% Use upquote if available, for straight quotes in verbatim environments
%§\IfFileExists{upquote.sty}{\usepackage{upquote}}{}
%§\IfFileExists{microtype.sty}{% use microtype if available
%§  \usepackage[]{microtype}
%§  \UseMicrotypeSet[protrusion]{basicmath} % disable protrusion for tt fonts
%§}{}
%§\makeatletter
%§\@ifundefined{KOMAClassName}{% if non-KOMA class
%§  \IfFileExists{parskip.sty}{%
%§    \usepackage{parskip}
%§  }{% else
%§    \setlength{\parindent}{0pt}
%§    \setlength{\parskip}{6pt plus 2pt minus 1pt}}
%§}{% if KOMA class
%§  \KOMAoptions{parskip=half}}
%§\makeatother
%§\usepackage{xcolor}
%§\IfFileExists{xurl.sty}{\usepackage{xurl}}{} % add URL line breaks if available
%§\IfFileExists{bookmark.sty}{\usepackage{bookmark}}{\usepackage{hyperref}}
%§\hypersetup{
%§  hidelinks,
%§  pdfcreator={LaTeX via pandoc}}
%§\urlstyle{same} % disable monospaced font for URLs
%§\setlength{\emergencystretch}{3em} % prevent overfull lines
%§\providecommand{\tightlist}{%
%§  \setlength{\itemsep}{0pt}\setlength{\parskip}{0pt}}
%§\setcounter{secnumdepth}{-\maxdimen} % remove section numbering
%§
%§\author{}
%§\date{}
%§
%§\begin{document}

\emph{In this manuscript, Sierra et al.~proposed that carbon allocation
should be from GPP instead of NPP in ecosystem models since NPP-based
models assumed autotrophic respiration only consumed fixed carbon
immediately without transit time. They introduced the conceptual
development of carbon allocation from NPP, reviewed 18 ecosystem models,
and analyzed the distributions of carbon transit time between the two
types of modeling schemes. The authors also showed that the NPP-based
model conflicted with empirical evidence on plants' age of respired
carbon. Overall, I enjoyed reading this paper. The logic of the
manuscript is clear, and the presentation of the ideas and perspectives
is precise and neat. I agree with the authors that there are many
advantages to modeling carbon allocation from GPP. From a modeler's
perspective, I found that some statements in the current version of the
manuscript may need more discussion. From the perspective of an
empirical ecologist, there are still some limitations to using GPP
instead of NPP for carbon allocation. Please find my major and specific
comments below.}

We thank the reviewer for the accurate description of our manuscript. We
followed these recommendations in a revised version, trying to balance
the modeling challenges with the available observations.

\emph{Major comments:} \emph{GPP and carbon allocation are simulated
with different time steps in many process-based ecosystem models. For
example, GPP is commonly simulated during the daytime with a time step
from half to three hours, but carbon allocation is updated daily. If the
models adopted the GPP-based carbon allocation scheme, they have to
improve the representations of diurnal changes in many processes related
to plant growth and carbon allocation. These improvements could
dramatically increase the complexity of the canopy module in the model.
Some additional discussions on this issue could help modelers better
understand the challenge of adopting the GPP-based carbon allocation
scheme.}

Thanks for the suggestion. We mentioned the differences in time-steps
among the different processes involved, but did not mention the
challenges for representing allocation at a timescale compatible with
the representation of GPP. To address this issue, we added a paragraph
in Section 2.2 adding more detail on these time-step differences, and
discussed this challenge in the recommendations section.

\emph{Figure 1 showed a constant Ra/GPP ratio in most CMIP Earth system
models. This pattern has also been reported in terrestrial ecosystem
models (e.g., https://doi.org/10.1002/2016JG003384). As mentioned by the
authors, a critical question is how to improve the modeling of
autotrophic respiration in the models. Unlike leaf photosynthesis, the
Ra scheme varies greatly among current ecosystem models. For example, in
the CLM4.5 model, the growth respiration (Rg) is calculated as a factor
of the total carbon in new growth on a given timestep, based on
construction costs for a range of woody and non-woody tissues. The
maintenance respiration (Rm) in CLM4.5 is a temperature function based
on a base rate of Rm. However, in the JULES model, Rm is simulated from
a moisture and nitrogen function based on dark leaf respiration. Rg in
JULES is further calculated as a fraction of the difference between GPP
and Rm. The authors have reviewed Rm in different models in section 2.1.
It would be better if they could provide some details of the modeling of
Ra in some specific models.}

Reviewer 1 correctly pointed out that the pattern presented in Figure 1
emerges by aggregating the data globally, but in fact most models
represent a dynamic allocation pattern. However, we would like to point
out that the emphasis of our analysis is not that allocation is a
constant proportion of GPP. This has been shown in other analyses as in
the reference provided by Reviewer 2. The emphasis of our analysis is in
the source of the carbon used for allocation, independent of whether
allocation is a constant fraction of GPP. Therefore, we decided to
reduce the emphasis in the text on this aspect, removed Figure 1, and
give now more relevance to the source of carbon for allocation.

We like the suggestion of the reviewer of giving more detail on how some
models represent carbon allocation. Although we do not want to give the
impression that one or two models are representative of the allocation
schemes of all models, we try to give some more details on CLM4.5 as
suggested by the reviewer.

\emph{I agree with the authors that ecosystem models need to incorporate
the non-structural carbon (NSC) pool dynamics. Adding the NSC pool into
the equation (1) or (2) could affect the solution of carbon transit time
because it changed the pool-flux structure in fig.~2. I'd like to
suggest the authors discuss whether and how adding the NSC pool can
influence the distributions of carbon transit time. Also, if we have
enough data to parameterize the age of the NSC pool in the models?}

As suggested by the reviewer, we added a section discussing in more
detail the addition of a non-structural carbon pool. In particular, we
make reference to the ACGCA model of Ogle and Pacala (2009), as well as
some of our previous work (Trumbore et al.~2017, Ceballos et al.~2018,
Herrera et al.~2020) where we modeled a NSC pool with an associated
distribution of carbon age. This previous work shows that NSCs used for
allocation and respiration can have ages as old as a decade. The NSC
reserves can support metabolism during periods of no or little
photosynthetic inputs, so they play an important role for representing
potential threats due to droughts and extreme events. The distribution
of ages for the NSC compartment helps to explain observed ages in
respired CO2, and can be used to constrain parameters of specific
allocation functions in models. However, the number of available
observations of ages in NSCs and respired CO2 are still very low, but we
hope that more studies will report them in the future, hopefully
motivated by what we discuss in this manuscript.

\emph{There are some benefits to using NPP-based carbon allocation,
especially in global models. First, the NPP-based scheme consists of
more measurable parameters than the GPP-based scheme. The increasing
observations of plant traits can be helpful in constraining those
parameters. This advantage could be important for those non-woody
ecosystems, in which the carbon allocation can be approximated by the
annual growth of different plant tissues. Second, because the GPP-based
scheme may need to increase the complexity of the canopy process, the
computation cost could increase dramatically for data assimilation.
Third, GPP itself is unmeasurable, so that the GPP uncertainty could
propagate to the carbon allocation.}

Reviewer 1 had a similar opinion on the measurability of GPP and the
increase in complexity in GPP-based schemes. Again, we believe both GPP
and NPP are extremely difficult to measure and there is no single
approach that can provide accurate estimates of both variables.
Available measurements of NPP only make a crude approximation of the
many different components that are often not measured (Clark et al
2001). With the new generation of GPP products such as those provided by
Fluxcom, a wealth of information is now available on GPP at the global
scale. Nevertheless, we understand the concern that many available
observations on traits may not be used if there is a change to
allocation based on GPP. We think this concern can be reduced if traits
such as allometric relations are used in the development of new
allocation functions. For instance, the Allometrically Constrained
Growth and Carbon Allocation (ACGCA) model proposed by Ogle \& Pacala
(2009) allocates carbon from GPP, has NSC pools to store carbon before
allocation and respiration, and uses allometric constrains to obtain
biomass growth. This modeling approach, or a variant of it, would be a
good candidate for implementing in new GPP-based schemes, and does not
imply aadditional complexity in modeling canopy processes.

\emph{I also agree with the authors that radiocarbon data is helpful for
improving the model. However, the measurements of radiocarbon are
expensive in many countries. Maybe some introductions or discussions of
available radiocarbon data from the ISRaD database are helpful for the
readers.}

We added more discussion on the use of radiocarbon. Although radiocarbon
measurements are indeed expensive, they are now being collected more
often and in many more studies as previously. One main development in
the previous years has been the introduction of the Mini radioCarbon
Dating System (Micadas), which has reduced costs and increased sample
throughput in many laboratories around the world. Unfortunately, the
International Soil Radiocarbon Database (ISRaD) only contains
radiocarbon measurements from soils, and does not include measurements
from plant parts and their respiration flux. However, there is potential
that in the future a radiocarbon database for plants becomes available.

\emph{Minor comments:} \emph{(1) It is better to give basic information
about the function $f_a(\tau)$ as described in Metzler et al.~(2018). Some
new readers could be unfamiliar with the matrix equation and its
solution.}

We added more details on this equation as suggested by the reviewer.

\emph{(2) Fig.1: Please add a few sentences to briefly describe those
Earth system models.}

We removed this figure from the manuscript, so model descriptions are no
longer needed.

\emph{(3) P13, L253-254: This statement might be too strong.}

It may be strong but it is true. An NPP-based model cannot use
radiocarbon data from respiration for model-data assimilation.

\emph{(4) The word ``model'' is used in different ways in the main text,
such as ecosystem model, ecosystem carbon model, coupled carbon-climate
model, land-surface model, carbon allocation model, etc. It is better to
reduce the diversity of model types in the text.}

Thanks for the suggestion. We adopted only one single description of
model, ecosystem model, and made more specific descriptions according to
the context.

%\end{document}
